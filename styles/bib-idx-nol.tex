%%%%%%%%%%%%%%%%%%%% 参考文献宏包 %%%%%%%%%%%%%%%%%%%%
%%参考文献编译方式:先用XeLaTeX编译主文件.tex一次;再用BibTeX编译主文件(或各子文件).tex一次;然后用XeLaTeX编译主文件.tex两次%%
\RequirePackage[numbers,sectionbib]{natbib}%参考文献格式
\renewcommand{\bibnumfmt}[1]{\protect\NumberedLabel{#1}}%修改参考文献序号样式
\renewcommand{\citenumfont}[1]{\color{ColorA}\CiteNumFont#1}%修改引用样式
\bibpunct{\color{ColorA}{[}}{\color{ColorA}{]}}{,}{n}{}{;}%修改方括号颜色
\renewcommand{\bibname}{参考文献}%修改参考文献标题
%%根据类布尔参数splitbib的存在与否来判断是否加载宏包chapterbib%%
\makeatletter
\ifTextBook@splitbib
\RequirePackage{chapterbib}%分章节显示参考文献,且参考文献编号各自独立
\fi
%%设置参考文献输出命令键%%
\define@boolkey{printbib}{intoc}[true]{}
\def\printbib@biblevel{\@ifundefined{chapter}{\section}{\chapter}*}
\define@key{printbib}{level}{\def\printbib@biblevel{#1}}
\def\printbib@toclevel{\@ifundefined{chapter}{section}{chapter}}
\define@key{printbib}{toclevel}{\def\printbib@toclevel{#1}}

%%设置默认键值%%
\setkeys{printbib}{level=\chapter*,toclevel=chapter}
%%参考文献样式设置命令%%
\newcommand{\bibsetup}[1]{\setkeys{printbib}{#1}}
%%参考文献输出设置%%
\newcommand{\printbib}[1]{
	\colorlet{ListColor}{ColorA}
	\renewcommand{\bibsection}{
		\printbib@biblevel{\bibname}
		\ifKV@printbib@intoc\phantomsection
		\addcontentsline{toc}{\printbib@toclevel}{\bibname}\fi}
	\bibliographystyle{plain}
	\bibliography{#1}}
\makeatother

%%%%%%%%%%%%%%%%%%%% 索引宏包 %%%%%%%%%%%%%%%%%%%%
\RequirePackage{filecontents}%在tex内编写数据文件
\RequirePackage[splitindex]{imakeidx}%索引设置,splitindex参数自动分割索引\\Large
%%索引风格文件indexstyle.ist%%
\begin{filecontents}[overwrite]{indexstyle.ist}
	headings_flag 1
	heading_prefix "\n\\centering\\bfseries\\color{ColorA}"
	heading_suffix "\\normalfont\\color{black}\\nopagebreak\n"
	delim_0 " \\hfill " 
	delim_1 " \\hfill "
	delim_2 " \\hfill "
\end{filecontents}

%%设置默认键值(不要引入noclearpage选项,否则会出现目录页码错误)%%
\setkeys{imkiindex}{level=\chapter,toclevel=chapter}

%\makeatletter
%%%添加新关键字envlevel(环境中索引层次)%%
%\def\imki@envindexlevel{\@ifundefined{chapter}{\section}{\chapter}*}
%\define@key{imkiindex}{envlevel}{\def\imki@envindexlevel{#1}}
%%%添加新关键字envtoclevel(环境中索引的目录层次)%%
%\def\imki@envtoclevel{\@ifundefined{chapter}{section}{chapter}}
%\define@key{imkiindex}{envtoclevel}{\def\imki@envtoclevel{#1}}
%%%设置默认键值(不要引入noclearpage选项,否则会出现目录页码错误)%%
%\setkeys{imkiindex}{level=\chapter*,envlevel=\chapter,toclevel=chapter,envtoclevel=chapter}
%%%修改索引环境%%
%\ifimki@original
%\expandafter\def\expandafter\theindex\expandafter{\expandafter
	%\imki@maybeaddtotoc\theindex}
%\else
%\global\let\imki@idxprologue\relax
%\renewenvironment{theindex}{
	%%%%%%%%%%% 修改部分开始 %%%%%%%%%%
	%\if@appendix%判断索引所在环境
	%\imki@envindexlevel{\indexname}
	%\else
	%\imki@indexlevel{\indexname}
	%\fi
	%\imki@maybeaddtotoc%添加目录
	%%\imki@indexheaders%注释掉页眉设置
	%%\thispagestyle{\imki@firstpagestyle}%注释掉首页页面样式
	%%%%%%%%%%% 修改部分结束 %%%%%%%%%%
	%\ifnum\imki@columns>\@ne
	%\columnsep\imki@columnsep
	%\ifx\imki@idxprologue\relax
	%\begin{multicols}{\imki@columns}
	%\else
	%\begin{multicols}{\imki@columns}[\imki@idxprologue]
	%\fi
	%\else
	%\imki@idxprologue
	%\fi
	%\global\let\imki@idxprologue\relax
	%\parindent\z@
	%\parskip\z@\@plus0.3\p@\relax
	%\columnseprule
	%\ifKV@imki@columnseprule0.4\p@\else\z@\fi
	%\raggedright
	%\let\item\@idxitem
	%\imki@othercode}{
	%\ifnum\imki@columns>\@ne
	%\end{multicols}
	%\fi}
%\fi
%
%%%修改目录添加%%
%\def\imki@putindexsplit#1{
	%\ifimki@nonewpage\else
	%\imki@clearpage
	%\ifimki@disableautomatic\else
	%\immediate\closeout\csname #1@idxfile\endcsname
	%\fi\fi
	%\let\imki@indexname\indexname
	%\@nameuse{imki@set@#1}\imki@decide
	%\if@tempswa
	%\imki@exec{\imki@program\imki@options#1.idx}
	%\else
	%\imki@finalmessage{#1}
	%\fi
	%%%%%%%%%%% 修改部分开始 %%%%%%%%%%
	%\ifKV@imki@intoc
	%\if@appendix
	%\def\imki@maybeaddtotoc{\@nameuse{phantomsection}
		%\addcontentsline{toc}{\imki@envtoclevel}{\imki@title}}
	%\else
	%\def\imki@maybeaddtotoc{\@nameuse{phantomsection}
		%\addcontentsline{toc}{\imki@toclevel}{\imki@title}}
	%\fi
	%\else
	%\def\imki@maybeaddtotoc{}
	%\fi
	%%%%%%%%%%% 修改部分结束 %%%%%%%%%%
	%\ifx\imki@title\imki@check@indexname\else
	%\def\indexname{\imki@title}
	%\fi
	%\@input@{#1.ind}
	%\let\indexname\imki@indexname}
%%%%%
%\def\imki@putindexunique#1{
	%\ifimki@nonewpage\else
	%\imki@clearpage
	%\fi
	%\let\imki@indexname\indexname
	%\@nameuse{imki@set@#1}\imki@decide
	%\if@tempswa
	%\ifimki@splitdone\else
	%\ifimki@disableautomatic\else
	%\immediate\closeout\@indexfile
	%\fi
	%\imki@exec{splitindex \imki@splitindexoptions\space\imki@jobname.idx}
	%\global\imki@splitdonetrue
	%\fi
	%\else
	%\ifimki@splitdone\else
	%\imki@splitindexmessage\global\imki@splitdonetrue
	%\fi\fi
	%\if@tempswa
	%\imki@exec{\imki@program\imki@options\imki@jobname-#1.idx}
	%\fi
	%%%%%%%%%%% 修改部分开始 %%%%%%%%%%
	%\ifKV@imki@intoc
	%\if@appendix
	%\def\imki@maybeaddtotoc{\@nameuse{phantomsection}
		%\addcontentsline{toc}{\imki@envtoclevel}{\imki@title}}
	%\else
	%\def\imki@maybeaddtotoc{\@nameuse{phantomsection}
		%\addcontentsline{toc}{\imki@toclevel}{\imki@title}}
	%\fi
	%\else
	%\def\imki@maybeaddtotoc{}
	%\fi
	%%%%%%%%%%% 修改部分结束 %%%%%%%%%%
	%\ifx\imki@title\imki@check@indexname\else
	%\def\indexname{\imki@title}
	%\fi
	%\@input@{\imki@jobname-#1.ind}
	%\let\indexname\imki@indexname}
%
%\makeatother

%%定义中文索引命令,#2是中文,#3是全拼(首字母)%%
\NewDocumentCommand{\zhindex}{omm}{\IfValueTF{#1}{\index[#1]{#3@#2}}{\index{#3@#2}}}

%%%%%%%%%%%%%%%%%%%% 术语表宏包 %%%%%%%%%%%%%%%%%%%%
\RequirePackage{nomencl}%引入术语表
%%生成主要符号表的方法:主文件先用XeLaTeX编译一次,然后建立一个.bat文件,内容为:makeindex <filename>.nlo -s nomencl.ist -o <filename>.nls -t <filename>.nlg,运行,再用XeLaTeX编译主文件两次即可生成%%
\renewcommand{\nomname}{主要符号表}%更改符号表标题

\makeatletter
%%以下命令会令nomencl宏包选项失效,符号表的具体设置需要在主文件中进行%%
%%设置符号表输出命令键%%
\newif\if@nomencl@columnseprule
\define@key{nomen}{columns}{\def\@nomencl@columns{#1}}
\define@key{nomen}{columnsep}{\def\@nomencl@columnsep{#1}}
%%添加新关键字level%%
\def\@nomencl@level{\@ifundefined{chapter}{\section}{\chapter}*}
\define@key{nomen}{level}{\def\@nomencl@level{#1}}
%%添加新关键字toclevel%%
\def\@nomencl@toclevel{\@ifundefined{chapter}{section}{chapter}}
\define@key{nomen}{toclevel}{\def\@nomencl@toclevel{#1}}
%%定义产生一对布尔键的命令%%
\newcommand{\@nomen@define@pairboolkeys}[4]{
	\define@boolkey{nomen}{#1}[true]{#2}
	\define@boolkey{nomen}{#3}[true]{#4}}
%%
\@nomen@define@pairboolkeys{columnseprule}{\@nomencl@columnsepruletrue}{nocolumnseprule}{\@nomencl@columnseprulefalse}
\@nomen@define@pairboolkeys{intoc}{\@intoctrue}{notintoc}{\@intocfalse}
\@nomen@define@pairboolkeys{tocbasic}{\@nomencl@tocbasictrue}{notocbasic}{\@nomencl@tocbasicfalse}
%%设置默认键值%%
\setkeys{nomen}{level=\chapter*,envlevel=\chapter,toclevel=chapter,envtoclevel=chapter,
	columns=2,columnsep=2.0em,notintoc,notocbasic,nocolumnseprule}
%%定义符号表设置命令%%
\newcommand{\nomensetup}[1]{\setkeys{nomen}{#1}}
%%修改符号表环境%%
\renewcommand{\thenomenclature}{
	%%%%%%%%%% 修改部分开始 %%%%%%%%%%
	\clearpage
	\providecommand*{\listofnlsname}{\nomname}
	\if@nomencl@tocbasic
	\let\list@fname\listofnlsname
	\def\@currext{nls}
	\tocbasic@listhead{\list@fname}
	\else
	\if@appendix%判断索引所在环境
	\@nomencl@envlevel{\nomname}
	\if@intoc%判断是否存在intoc参数
	\phantomsection
	\addcontentsline{toc}{\@nomencl@envtoclevel}{\nomname}
	\fi
	\else
	\@nomencl@level{\nomname}
	\if@intoc%判断是否存在intoc参数
	\phantomsection
	\addcontentsline{toc}{\@nomencl@toclevel}{\nomname}
	\fi\fi
	\fi
	\nompreamble
	\ifnum\@nomencl@columns>\@ne%多列符号表设置
	\begin{multicols}{\@nomencl@columns}
		\columnsep\@nomencl@columnsep
		\columnseprule
		\if@nomencl@columnseprule0.4\p@\else\z@\fi
		\fi
		%%%%%%%%%% 修改部分结束 %%%%%%%%%%
		\if@nomentbl
		\let\itemOrig=\item
		\def\item{\gdef\item{\\}}
		\expandafter\longtable\expandafter{\@nomtableformat}
		\else
		\list{}{
			\labelwidth\nom@tempdim
			\leftmargin\labelwidth
			\advance\leftmargin\labelsep
			\itemsep\nomitemsep
			\let\makelabel\nomlabel}
		\fi}
	%%
	\def\endthenomenclature{
		\if@nomentbl
		\item\endlongtable
		\global\let\item=\itemOrig
		\else
		\endlist
		\fi
		%%%%%%%%%% 修改部分开始 %%%%%%%%%%
		\nompostamble
		\ifnum\@nomencl@columns>\@ne
	\end{multicols}
	\fi
	%%%%%%%%%% 修改部分结束 %%%%%%%%%%
}
\makeatother