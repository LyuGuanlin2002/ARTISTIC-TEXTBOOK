\documentclass[themecolor=iris]{textbook-cn}
\graphicspath{{./figure/}{./figures/}{./image/}{./images/}{./graphics/}{./graphic/}{./pictures/}{./picture/}}%提供多种图片路径
\input{./styles/Cover-cn.tex}
\input{./styles/Logo.tex}
\input{./styles/Choice.tex}

\usepackage{zhlipsum,lipsum}%乱数假文(测试文字)
\usepackage{wrapfig}%文字环绕图片






%%设置书籍信息%%
\seriesname{系列名称}
\serieslogo{Springer.pdf}
\version{一}
\title{标题文字}
\englishtitle{English Title}
\subtitle{副标题文字}
\englishsubtitle{English Subtitle}
\author{作者甲\\作者乙\\作者丙\\扬斯克}
\pressname{出版社}
\presslogo{bookpresslogo.pdf}


%%logo插入命令%%
\newcommand{\inputlogo}[1]{\raisebox{-0.5em}{\includegraphics[width=1.5cm]{#1}}~~}
\newcommand{\Hinputlogo}[2]{\href{#2}{\inputlogo{#1}}}

%%图片形状裁剪命令%%
\newcommand{\cutimage}[4]{%
\tikz{%
\clip[rounded corners=2.5mm](-\linewidth/2,-\linewidth/2)rectangle(\linewidth/2,\linewidth/2);
\node at(#2,#3){\includegraphics[width=#1]{#4}};}%
}





%%%%%%%%%%%%%%%%%%%%%%%%%%%%%%%%%%%%%%%%%%%%%%%%%%%%%%%%%%%%%%%%%%%%%%%%%%%%%%%%%%%
\begin{document}

%%标题页(扉页)
\maketitle


\zhlipsum[1]
\lipsum[1-6]



\frontmatter
\lipsum



\chapterimage{\includegraphics[height=\paperheight]{parasol.jpg}}
\chapter{测试文字}

\begin{figure}
	\centering
	\begin{tikzpicture}[>=stealth, thick]
		\draw (0,0)--(2,0)--(2,1)--(0,1)--(0,0);
		\fill [pattern = north east lines] (-1,-.75) rectangle (3,-.5);
		\draw(-1,-.5)--(3,-.5);
		\draw (.5,-.25) circle (.225);
		\draw (1.5,-.25) circle (.225);
		\draw (.5,-.25) circle (.1);
		\draw (1.5,-.25) circle (.1);
		\draw[->, dashed](3,0.5)--(-2,0.5);
		\draw[very thick](0,.5)--(-2,0.5);
		\foreach \i in {-.4,-.8,-1.2,-1.6,-2}
		{
			\draw (\i,0.6)--(\i,.5);
		}
		\draw (-1,1)--(-1,1.1);
		\draw (-0.6,1)--(-0.6,1.1);
		\draw[-] (-1,1)--(-0.6,1);
		\node at (-0.8,1.35){$20$N};
	\end{tikzpicture}
	\caption{图中的虚线表示力的作用线}
\end{figure}




%%序言
\chapterimage{\includegraphics[height=\paperheight]{parasol.jpg}}
\chapter{前言}
\zhlipsum
\lipsum

%%小目录
\shorttableofcontents


%%目录
\tableofcontents*[2][\noindent%
\cutimage{1.5\linewidth}{-3.0mm}{0}{iris.jpg}
\vspace*{3.0mm}\\
\cutimage{1.5\linewidth}{3.0mm}{0}{starry.jpg}
\vspace*{3.0mm}\\
\cutimage{1.5\linewidth}{0}{0}{vineyard.jpg}
\vspace*{3.0mm}\\
\cutimage{1.5\linewidth}{0}{0}{olive.jpg}
\vspace*{3.0mm}\\
\cutimage{1.5\linewidth}{0}{0}{rhone.jpg}
\vspace*{3.0mm}\\
\cutimage{1.5\linewidth}{0}{0}{harvest.jpg}
\vspace*{3.0mm}\\
\cutimage{\linewidth}{0}{0}{sunflower.jpg}
\vspace*{3.0mm}\\
\cutimage{\linewidth}{0}{0}{iris1.jpg}
\vspace*{3.0mm}\\
\cutimage{\linewidth}{0}{-5.0mm}{cypresses.jpg}
\vspace*{3.0mm}\\
\cutimage{2.5\linewidth}{28.0mm}{0}{thunderclouds.jpg}
\vspace*{3.0mm}\\
\cutimage{1.5\linewidth}{0}{0}{poppies.jpg}
\vspace*{3.0mm}\\
\cutimage{1.5\linewidth}{0}{0}{clifftop.jpg}
\vspace*{3.0mm}\\
\cutimage{1.5\linewidth}{0}{0}{garden.jpg}
\vspace*{3.0mm}\\
\cutimage{\linewidth}{0}{-4.0mm}{parasol.jpg}
]

%%图片列表
\listoffigures





\mainmatter


\partintro{\lipsum[2][1-5]}
\partimage{\includegraphics[height=\paperheight]{blossom.jpg}}
\part[短标题]{力学}
\lipsum


\begin{figure}
	\centering
	\begin{tikzpicture}[>=stealth, thick]
		\draw (0,0)--(2,0)--(2,1)--(0,1)--(0,0);
		\fill [pattern = north east lines] (-1,-.75) rectangle (3,-.5);
		\draw(-1,-.5)--(3,-.5);
		\draw (.5,-.25) circle (.225);
		\draw (1.5,-.25) circle (.225);
		\draw (.5,-.25) circle (.1);
		\draw (1.5,-.25) circle (.1);
		\draw[->, dashed](3,0.5)--(-2,0.5);
		\draw[very thick](0,.5)--(-2,0.5);
		\foreach \i in {-.4,-.8,-1.2,-1.6,-2}
		{
			\draw (\i,0.6)--(\i,.5);
		}
		\draw (-1,1)--(-1,1.1);
		\draw (-0.6,1)--(-0.6,1.1);
		\draw[-] (-1,1)--(-0.6,1);
		\node at (-0.8,1.35){$20$N};
	\end{tikzpicture}
	\caption{图中的虚线表示力的作用线}
\end{figure}




%%第一章
\include{M1.tex}





\chapterimage{\includegraphics[height=\paperheight]{harvest.jpg}}
\chapter{测试测试测试测试测试测试}

{\FontSizeSet[3.0]{14pt}
\zhlipsum[1]

}

\backmatter

\chaptersubtitle{测试文字}
\chapter{后记}
\footnote{测试脚注测试脚注}
\lipsum
\footnote{测试脚注测试脚注}

\chapter{测试文字}
\zhlipsum

\cleardoublepage

\lipsum


\chaptersaying{物华天宝,龙光射牛斗之墟;人杰地灵,徐孺下陈蕃之榻。}
\chapter{测试文字}








{
\FontSizeSet[1.0]{14pt}
\zhlipsum[2]
}

\partsubtitle{副标题测试副标题测试副标题测试副标题测试}
\partintro{\lipsum[2]}
\part*{测试文字测试文字}

\zhlipsum



\partsubtitle{副标题测试副标题测试副标题测试副标题测试}
\partsubtitle{ }
\partintro{\lipsum[2]}
\part*{测试文字测试文字}

\zhlipsum


我一定会让你 \leaders\hrule height -2.6pt depth 3pt \hskip 3.0cm 满意的。






\end{document}



















